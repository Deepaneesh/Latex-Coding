\documentclass[a4paper,12pt]{article}
\title{Documentation using \LaTeX}
\author{Deepaneesh R V }
\date{03-07-2024}
\begin{document}
	\maketitle
	Hello everyone ! , \\
	i am so glad to use \LaTeX for my docmentation process.\\
	we will see how to use latex for documentation in our below sections .
	
	\section{how to use \LaTeX}
	\subsection{To make in bold}
	if use are interested to print the word "word" in bold use "/textbf\{\}" for making it bold ,\\
	\textbf{for example :} "word " is normal , \textbf{"word"} is in bold
	\subsection{To make the font tyle in italic}
	if use are interested to print a word "word" in italic font use "/textit\{\}" for making it in italic
	,\\
	\textbf{for example :} "word " is normal , \textit{"word"} in italic
	\section{To use special character}
	if use are interested to print any special character like "\$","\%","\&",etc..... \\
	there is some rule in \LaTeX code .\\
	the code is so simple just add the \textbf{left- slash } before your symbol
	\pagebreak
	\section{Adjusting the font size}
	there are diffent types of font size in \LaTeX there are \\
	\textbf{"tiny","scriptsize","footnotesize","small","normalsize","large","LARGE",\\"huge","Hu
		ge"}
	these are the font size in acsending order \\
	\textbf{for example} \tiny hello , \scriptsize hello, \footnotesize hello, \normalsize hello ,
	\large hello ,
	\LARGE hello , \huge hello , \Huge {hello }
	\section{Allignment in paragraph}
	
	\subsection{Left Allignment }
	\begin{flushleft}
		\small
		Hello everyone ! , \\
		i am so glad to use \LaTeX for my docmentation process.\\
		we will see how to use latex for documentation in our below sections .
	\end{flushleft}
	\subsection{Right agglignment}
	\begin{flushright}
		\large
		Hello everyone ! , \\
		i am so glad to use \LaTeX for my docmentation process.\\
		we will see how to use latex for documentation in our below sections .
		
	\end{flushright}
	
	\pagebreak
	\subsection{ CenterAllignment }
	\begin{center}
		\LARGE
		Hello everyone ! , \\
		i am so glad to use \LaTeX for my docmentation process.\\
		we will see how to use latex for documentation in our below sections .
		
	\end{center}
	\section{Displaying quotes}
	\large
	\ Elon musk said;
	\begin{quote}
		Narrow AI \& General AI is ok , but Super AI is risk!
	\end{quote}
	
\end{document}