\documentclass[12pt,a4paper]{article}
\begin{document}
	
	\section*{Scope of Statistics and Types of Data}
	In this unit, we present the meaning of statistics, various definitions, origin and
	growth, functions, scope and applications to different fields such as Agriculture, Economics
	and many more. We also define ‘data’ , various types of data and their measurement scales.
	\begin{itemize}
		\item Origin and Growth of Statistics
		\item Definitions
		\item Measurement Scales
	\end{itemize}
	\subsection*{Origin and Growth of Statistics}
	\begin{itemize}
		\item The origin of statistics can be traced back to the primitive man, who put notches on trees to
		keep an account of his belongings.
		\item In olden days statistics was used for political- war purpose.
	\end{itemize}
	\subsection*{Definitions}
	\begin{enumerate}
		\item‘Statistics is the science of counting’ -A. L .Bowley
		\item Wallist and Roberts defines statistics as “Statistics is a body of methods for making decisions in
		the face of uncertainty”
	\end{enumerate}
	\subsection*{Measurement Scales}
	There are four types of data or measurements scales called nominal, ordinal, interval
	and ratio. These measurement scale is made by Stanley Stevens.
	\begin{description}
		\item [Nominal scales:] They are numerical for name sake only.
		\item[Ordinal scales:] The order of the values is important but the differences between each one is
		unknown.
	\end{description}
	
\end{document}